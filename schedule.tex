\documentclass{standalone}

%\usepackage[margin=1in]{geometry}
\usepackage{longtable}
\usepackage{array}
\usepackage[table]{xcolor}
\usepackage{graphicx}
\pagenumbering{gobble}

\begin{document}

\renewcommand{\arraystretch}{1.4}

\begin{tabular}{|>{\raggedright}p{2.5cm}|>{\raggedright}p{5cm}|>{\raggedright}p{5cm}|>{\raggedright\arraybackslash}p{3cm}|}
\hline
\textbf{Week} & \textbf{Monday Class} & \textbf{Wednesday Class} & \textbf{Suggested Readings} \\
\hline
\multicolumn{4}{|c|}{\cellcolor{gray!20}\textbf{Module 1: Course Introduction}} \\
\hline
25-27 August & Course Overview. \newline Skill Assessment (for Group Formation) & Introduction to PyTorch &
A. Chapter 12: {\small First steps with PyTorch}\\
\multicolumn{4}{|c|}{\cellcolor{gray!20}\textbf{Module 2: Dealing with High Dimensional Data}} \\
\hline
1-3 September & Labour Day (No class) & Refreshing Regularized Regression through PyTorch and Stochastic Gradient Descent  & \\
\hline
8-10 September & Principal Component Analysis as a preprocessing step. \newline Introduction to Pipelines & {\bf Lab}: Partial Least Squares Regression (PLS) & \\
\hline
\multicolumn{4}{|c|}{\cellcolor{gray!20}\textbf{Module 3: Exploiting Non-Linear Relationships in Data}} \\
\hline
15-17 September & Support Vector Machine I: The Linear Case & {\bf Lab}: Polynomial Regression as a First Approximation to Non-Linear problems & A. Chapter 3: {\small Maximum margin classification with support vector machines}\\
\hline
22-24 September & Support Vector Machine II: The Non-Linear case through the Kernel Trick & {\bf Lab}: Predicting Seismic Building Response with Support Vector Regrssion & A. Chapter 3: {\small Solving nonlinear problems using a kernel SVM}\\
\hline
29 September-1 October & Kernels are EVERYWHERE: Kernel PCA \& Kernel Ridge Regression,  & {\bf Lab}: Non-linear dimensionality reductions: t-SNE, and UMAP
 & \\
\hline
6-8 October & Neural Networks I: Introduction \& Multi-Layer Perceptrons (MLPs) & {\bf Lab}: Manual implementation of MLPs for Classification and Regression using toy data & A. Chapter 11: {\small  Modeling complex functions with artificial neural networks}\\
\hline
13-15 October & Reading Day (No Class) & {\bf Lab}: Application of Regularization to Survival Analysis & A. Chapter 12: {\small Building an NN model in PyTorch and Choosing activation functions for multilayer neural
networks}%. A. Chapter 13: {\small Making model building more flexible with nn.Module"}
\\
\hline
20-22 October & Neural Networks II: Deeper into model building, training and opmitization. & {\bf Lab}: Application of  Neural Networks to Real Data & \\
\hline
27-29 October & Neural Networks III: Improving generalization & {\bf Lab}: Implementing an autoencoder for non-linear dimensionality reduction  & \\
\hline
\multicolumn{4}{|c|}{\cellcolor{gray!20}\textbf{Module 4: Ensemble Methods \& Meta-Learning}} \\
\hline
3-5 November & Bagging: Bootstrap Aggregating and Random Forest & Boosting: Adaptive Boosting and Gradient Boosting & \\
\hline
10-12 November & {\bf Lab}: Classifying Internet of Things Device Types with Bagging and Boosting Methods  & {\bf Lab}:  Improving Student Performance Prediction through Stacking & \\
\hline
\multicolumn{4}{|c|}{\cellcolor{gray!20}\textbf{Module 5: Tackling Population Heterogeneity}} \\
\hline
17-19 November & Clustering I: k-Means & Clustering II: Hierarchical Clustering, Gaussian Mixture Models & \\
\hline
26-28 November & Clustering III: Validation &  Thanksgiving (No class) & \\
\hline
1-3 December & {\bf Lab}: Brain Co-activation Patterns using Clustering & \multicolumn{2}{c|}{Final Project Time} \\
\hline
8-10 December & \multicolumn{3}{c|}{Final Project Time} \\
\hline
\end{tabular}


\end{document}
